\section{Introduction}
\vspace{-0.2cm}

Experimental results indicate that connectivity in cortical circuits is highly nonrandom:
\begin{enumerate}[leftmargin=2cm]
  \itemsep0pt
\item Reciprocally connected pairs appear more often than expected by chance \cite{Song2005, Perin2011}
\item Certain triplet motifs of neurons are strongly overrepresented \cite{Song2005}
  \item Clusters of neurons show more often a high degree of connectivity than predicted from spatial connectivity \cite{Perin2011}
  \end{enumerate}

Recently, these characteristics have been found in detailed reconstructions of cortical circuits \cite{Gal2017} and it was shown that generic network models can account for some of these features \cite{vegue2017}. However, it remains unclear what properties of a cortical circuit affect the connectivity such that the structures above emerge. Here we propose a simple network model based on the stereotypical anisotropy of axon morphology (Fig.~\ref{fig:morphaniso}).

\begin{center}\vspace{0.01cm}
  \includegraphics[width=1.0\columnwidth]{%
    /home/fh/sci/rsc/aniso_netw/pub/arxiv18/figures/morphological_anisotropy/morphological_anisotropy_horizontal.pdf}
  \captionof{figure}{Projections of thick-tufted layer V pyramidal
cells in the rat somatosensory cortex (manually traced from \cite{Romand2011}) \textbf{A} Axon morphology of a single cell (left), branch density heatmap of axons for 5 neurons. \textbf{B} As in A, but for dendrite.}
  \label{fig:morphaniso}
\end{center}\vspace{2cm}

